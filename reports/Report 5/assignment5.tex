\documentclass[a4paper,12pt]{article}

\usepackage[a4paper]{geometry}

\usepackage[utf8]{inputenc}            % Use utf8 input encoding
%\usepackage[latin1]{inputenc}         % Use iso 8859-1 encoding
\usepackage[T1]{fontenc}               % T1 fonts (support for accents/diacritics)
\usepackage{lmodern}                   % font with proper T1 support and good glyph quality

\usepackage{listings}                  % for (code) listings
\usepackage{amsmath}                   % AMS math typesetting
\usepackage{titlesec}

\usepackage{hyperref}                  % better references for PDF

\titleformat{\section}{\LARGE\bfseries}% hide redundant number
            {}{0pt}{}

\begin{document}

\begin{center}
	\rule{\textwidth}{0.1pt}\\[1cm]
	
	\Large Softwarepraktikum SS 2021\\\bf Assignment 5 Report %EDIT
\end{center}


\begin{center}

	\rule{\textwidth}{0.1pt}\\[0.5cm]

	{\Large Group 3\\[5mm]} %EDIT

	\begin{tabular}{lll}
		%EDIT
		Jamie Anike Heikrodt & 394705 & anike.heikrodt@rwth-aachen.de \\

		Thomas Pollert & 406215 & thomas.pollert@rwth-aachen.de \\
		
		Jascha Austermann & 422571 & jascha.auster@rwth-aachen.de \\

		Silyu Li & 402523 & silyu.li@rwth-aachen.de \\

	\end{tabular}\\[0.5cm]

	\rule{\textwidth}{0.1pt}\\[1cm]
	
\end{center}

% Uncomment next two lines for table of contents
\newpage
\tableofcontents

%\newpage
Jamie worked on Task 1 for the last assignment already. This week she implemented the Aspiration window algorithm.\\
Thomas worked on finding a good windowsize and benchmarking the Aspiration window algorithm.
\section{Task 1}
Already implemented and discussed in the previous assignment.
\section{Exercise 2}
\subsection{Discussions}
\begin{itemize}
    \item How to represent the map, the different field and the transitions.
    \item How to efficiently calculate the neighbouring tiles of a tile.
\end{itemize}

\subsection{Solutions}
  \begin{itemize}
      \item In this task we used the \textbf{2-dimensional array} to represent the coordinate system of the map, which was the first idea came to our mind.
      \item Use \textbf{Hashmap} of type (x,y,direction) -> Tile(Neighbour) to store the extra transitions given by the server. 
      \item \textbf{First thought}: Define a class \textbf{Tile} with several subclasses to represent different types of tiles, which allows good modification possibilities.\\ \textbf{Improvement}: Use of enumeration class \textbf{TileTypes} combining with the class \textbf{Tile} with several methods and attributes.
      \item \textbf{First thought}: Calculate the neighbouring tiles by calculating the (x,y) coordinates of them, which would not work so well for special transitions.\\ \textbf{Second thought}: Store for each tile its 8 neighbours, which would cause too much memory overhead.\\ \textbf{Improvement}:Firstly check the Hashmap if there are special transitions, then calculate with (x,y) coordinates. 
  \end{itemize}

\section{Exercise 3}
\subsection{Discussions}
The third task deals with benchmarks. For this assignment we wanted to create a basic foundation for our benchmark strategy and use it to test and compare Minimax and AlphaBeta. 
\subsection{Solutions}
To benchmark our programm, we use \textit{System.currentTimeMillis()}. To measure the time methods take in our programm, we just calculate the distance between the time before and after the method call. Some results of this regarding MiniMax and AlphaBeta will be presented in our group-meeting.\\
Future benchmarks will require to test different parts of methods. We will use the same benchmark technique as for simple method calls. This requires making clones of our methods, but gives us the chance to benchmark most parts of our code easily.\\
The data obtained when benchmarking is written in a CSV-file and then plotted with "datplot" by Michael Vogt. This gives us the chance to write all obtained data of one benchmark in a file and plot by different values later.




\end{document}